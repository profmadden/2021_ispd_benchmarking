% The first command in your LaTeX source must be the \documentclass command.


\documentclass[sigconf]{acmart}
\settopmatter{printacmref=true}

\fancyhead{}
  % do not delete this code.

\usepackage{balance}
  % for creating a balanced last page (usually last page with references)

% defining the \BibTeX command - from Oren Patashnik's original BibTeX documentation.
\def\BibTeX{{\rm B\kern-.05em{\sc i\kern-.025em b}\kern-.08emT\kern-.1667em\lower.7ex\hbox{E}\kern-.125emX}}
    
% Rights management information. 
% This information is sent to you when you complete the rights form.
% These commands have SAMPLE values in them; it is your responsibility as an author to replace
% the commands and values with those provided to you when you complete the rights form.
%
% These commands are for a PROCEEDINGS abstract or paper.


% Submission ID. 
% Use this when submitting an article to a sponsored event. You'll receive a unique submission ID from the organizers
% of the event, and this ID should be used as the parameter to this command.
%\acmSubmissionID{123-A56-BU3}


% end of the preamble, start of the body of the document source.


\usepackage{algorithm}
\usepackage{algorithmic}
\usepackage{epsfig,url}
\usepackage{epsf}

%% Added by JJ
\usepackage{cleveref}
\usepackage{tabularx}
\usepackage{ragged2e}
%%

%Control space between caption and figure/table
\captionsetup[figure]{font=small,skip=5pt}
%\captionsetup[table]{font=small,skip=0pt}


\copyrightyear{2021}
\acmYear{2021}
\setcopyright{acmcopyright}
\acmConference[ISPD '21]{Proceedings of the 2021 International Symposium on Physical Design}{March 29-April 1, 2021}{Virtual} \acmBooktitle{Proceedings of the 2021 International Symposium on Physical Design (ISPD '21), March 29-April 1, 2021, Virtual}
\acmPrice{15.00} \acmDOI{10.1145/3372780.3375563} \acmISBN{978-1-4503-7091-2/20/03}

\begin{document}


\title{Still Benchmarking After All These Years}
\iffalse
\author{Blind Review}
\else
\author{Ismail S. K. Bustany}
\email{ISMAILB@xilinx.com}
\affiliation{
  \institution{Xilinx}
}
\author{Jinwook Jung}
\email{jinwookjung@ibm.com}
\affiliation{
  \institution{IBM T.J.\ Watson Research Center}
  \streetaddress{1101 Kitchawan Rd}
  \city{Yorktown Heights}
  \state{New York}
  \postcode{10598}
}
\author{Patrick H. Madden}
\email{pmadden@binghamton.edu}
\affiliation{
  \institution{SUNY Binghamton CSD}
  \streetaddress{Box 6000}
  \city{Binghamton}
  \state{New York}
  \postcode{13902}
}
\author{Natarajan Viswanathan}
\email{nviswan@cadence.com}
\affiliation{
  \institution{Cadence}
}
\author{Stephen Yang}
\email{stepheny@xilinx.com}
\affiliation{
  \institution{Xilinx}
}


 
\fi
\begin{abstract}
Circuit benchmarks for VLSI physical design
have been growing in size and complexity, helping
the industry tackle new problems and find new
approaches.  In this paper, we take a look back
at how benchmarking efforts have shaped the
research community, consider trade-offs that
have been made, and speculate on what may come
next.


\end{abstract}

\begin{CCSXML}
\end{CCSXML}

\ccsdesc[500]{Hardware~Placement}


%
% Keywords. The author(s) should pick words that accurately describe the work being
% presented. Separate the keywords with commas.

\keywords{benchmarking; circuit placement; metrics}

\maketitle

\section{Introduction}

Integrated circuits have evolved at a breakneck pace for decades.  The
first MOSFET transistors were created at the end of the 1950's, with
Moore\cite{Moore650114} making his bold prediction for exponential
growth in 1965.  While it might have seemed far-fetched or wildly
optimistic at the time, Moore's Law has been a relentless juggernaut.
In 1960, only a handful of transistors could be integrated into a
single device, but by 1980, just twenty years later, the Motorola
68000 had roughly sixty-eight thousand transistors.  Twenty years
after that, the Intel Pentium 4 featured roughly forty-two million
transistors.  Twenty years after that, in the era in which this paper
is being written, the Apple M1 processor has in excess of sixteen
billion transistors.

None of these advances have been spontaneous.  At each
step along the way, researchers from both academia and
industry have been quietly identifying and overcoming
technological barriers.  The massive electronics industry,
which has impacted almost every aspect of modern life,
is the product of decades of hard work.

Market forces have been driving the industry forward; there are
fortunes to be made in modern electonics.  These forces have also been
something of a barrier.  There is competitive advantage to secrecy
around process technologies, circuit design, and design methodology --
but this secrecy also makes it challenging for a broad community of
researchers to solve new problems.


In this paper, we look back on how progress has been made
in physical design, with an emphasis on circuit placement.
Benchmarks have been a critical part of enabling progress;
following something of a rocky start, the physical design
community has established a process that has taken the field
from entirely uncharted territory into a state where sustained,
regular progress can be made.


% https://en.wikipedia.org/wiki/Transistor_count
% Better citation for stuff like this?
\iffalse
{\bf How to say this?
  Economic driver pushing technology forward.  Lots of
  companies competing for market share.  Secrecy on
  design technology, circuit designs, gives a competitive
  advantage -- but also limits the number of people who
  can solve the problems that need to be solved.

  For academic researchers in particular to be helpful,
  details of the problems needing solving have to be
  available.  Without test cases, and things that are
  suitable for publication, they cannot assist.
}




Making progress in design automation has been challenging;
the problems addressed are fundamentally hard from a scientific
perspective. Complicating matters further are the financial
implications -- semiconductor manufacturers, design tool companies,
and circuit designers, are all interested in making profits, and
this depends on some degree of secrecy in their work.  At the
same time, solving design problems requires assistance from
others.
\fi


\section{Laying the Foundations}

The very earliest circuits were designed by hand, each one a custom
creation. As design sizes increased, and semiconductor fabrication
technologies became more refined, designers converged on a few
common conventions and abstractions, such as lambda-based design rules
and standard cell libraries\cite{Mead93}.  A design flow of logic
synthesis, to placement, and to routing, enabled interchangable
design automation tools, and common design file formats.

For the research community, however, there were few common benchmark
problems.  In 1987, a panel discussion\cite{Preas87} highlighted the
need for good benchmarks for standard cell design, and a few years
later in 1991, the MCNC benchmarks\cite{Kozminski91} were presented
and quickly gained prominence.  The benchmarks are shown in Table
\ref{tab:mcnc} -- while they were not comparable in size and
complexity to the leading edge of industrial design, they filled a
vacuum, and were a catalyst for academic groups.  The availability of
circuits that could be used as physical design benchmark sparked a
wave of circuit placement research.

\begin{table}
  \begin{tabular}{|l|r|r|r|} \hline
Name & \# cells & \# nets & \# I/O \\ \hline
    fract   &   125 &   147 & 24 \\ \hline
   primary1 &   752 &   985 & 81 \\ \hline
    struct  &  1888 &  1920 & 64 \\ \hline
  industry1 &  2271 & 2593 & 814 \\ \hline
  primary2 & 2907 & 3136 & 107 \\ \hline
  biomed & 6417 & 5742 & 97 \\ \hline
  industry2 & 12142 & 13419 & 495 \\ \hline
  industry3 & 15059 & 21940 & 375 \\ \hline
  \end{tabular}
  \caption{The original MCNC standard cell
    benchmarks\cite{Kozminski91}.}\label{tab:mcnc}
\end{table}


There was hope that more benchmarks would
become available\cite{Kozminski91},
but relatively little happened.
A decade later, the original MCNC
benchmarks were still in wide use, and
the gap between academic research and
industrial practice had widened.
Even more troubling -- to try to keep pace, many research
groups were scaling and translating the benchmarks in
a variety of ways, resulting in massive variations in how
results were reported\cite{Madden010030}.

\section{Placement Benchmarking}

A lack of new circuit benchmarks hampered the work of many academic
groups.  What was needed was a way to break the log-jam, and this
arrived in the form of the release of ``ISPD98'' partitioning
benchmarks by Alpert\cite{Alpert980080}.  The net lists in this set of
benchmarks were derived from industrial circuits, where net names and
the functionality of logic elements had been obscured.  This protected
the intellectual property of the original designs, while still providing
a realistic overall structure.  Using this as a starting point,
Wang\cite{Wang000260} created ``placement-like'' benchmarks by simply
mapping vertex weights to cell sizes.  Carrying the idea further,
designs with a mixture of macro blocks and cells were created by
Adya\cite{Adya020012}.


\begin{quote}
All information relating to circuit functionality, timing and technology is removed. Unfortunately, this limits the direct applicability of these circuits (e.g., functional rep- lication for partitioning); yet, the release of these circuits would have been impossible otherwise. \cite{Alpert980080}
\end{quote}

% industry practice\cite{Adya040472}.

After a number of years of little change, there was a burst of
activity, drawing in researchers from both academia and industry.  The
GSRC supported GTX project \cite{Caldwell000693} provided a platform
to work from.

% ISPD98 partitioning benchmarks by Chuck

% Partition to placement
% \cite{Wang000260}
% Converted to mixed size placement in ISPD02 by Igor
% \cite{Adya020012}
% Mixed size placer PHM in ISPD04
% \cite{Khatkhate040084}
% ISPD05 -- Adaptec and Bigblue, fixed macros
% \cite{Nam050216}


% \subsection{Bookshelf File Formats}

It's worth highlighting the file formats used for the new benchmarks.
Rather than utilizing industry standards such as LEF and DEF,
there was an emphasis on new easy-to-parse file formats as part
of the GSRC ``bookshelf''\cite{umichbookshelf}.

For academic research groups, and particularly for new graduate
students working in the area, industrial-grade files can
be overwhelming.  Simply developing a parser to handle these
formats is a massive undertaking; software libraries that can
parse the files require extensive call-back support.

By contrast, the Bookshelf formats require little more than
a ``for loop'' to parse effectively, freeing up academic researchers
to focus on new algorithms.  


% SIGDA newsletter! Franc Brglez from MCNC
% http://web.cs.ucla.edu/classes/layout/testing/Examples.iscas/ISCAS89/Benchmark.readme

% {\bf Talk about production file formats versus things that
%   are easy to parse.  Decisions made for GSRC Bookshelf.}

% \cite{Caldwell000693}\cite{umichbookshelf}

\subsection{Benchmarking Objectives}

Why we do benchmarking.  This is all stuff that we ``know,'' but
may not have said explicitly.  And often, what we assume that we
all ``know'' is not in fact common knowledge.


\subsubsection{Better Solutions for Problems}

Industry can highlight critical problems, ones where new solutions
are needed.

Talk about gap between academic and industry designs, lack of
public access to leading edge stuff.  Industry needs competitive
advantage, holds information back.  Academic groups, even
with access, may be unable to publish.


\subsubsection{A Training Ground for Researchers}

All innovation is the product of men and women, working in both
academia and industry, around the world.  They get old, retire,
move into new areas.  Need a constant stream of new people to
carry on the work.

Well designed benchmarks give clear, tractable goals, and can
bring new people into the field.

Graduate students learn about physical design, hone their software
development skills.


\subsubsection{A Platform for Experimentation}

Fast turn-around invites higher-risk approaches, new ideas.

If it's only a few weeks worth of work, can try crazy
ideas, do things the ``wrong'' way, and stumble upon new
methods.

\subsubsection{Building the Community}

Level playing field, independent evaluation of results,
both public and private benchmarks avoid overtuning.

Opportunity for students in different research groups
to meet and interact, develop friendships and opportunities
for future collaboration.



\section{The ISPD Contests}

Beginning in 2005, ISPD has run annual contests to evaluate
different approaches to key physical design problems.  For circuit
placement, the contsts have occurred in 2005, 2006, 2011, 2014,
2015, with 2016 and 2017 focusing on placement for
field programmable gate arrays.

{\em talk about each year of the contest.  Metrics.  Citations
  for important publications based off the contest benchmarks.}

\subsection{Mixed Size Placement, 2005, 2006}

{\bf Gi-Joon}.  
\cite{ISPD05_contest}
Details here about ISPD 2005.  Cite relevant papers.
Benchmarks are adaptec, bigblue versions.  Legal2.pl script to
check legality.  Nine teams competing.
Fixed macros, from 200k to 2.1 million cells.

2006 adds target density, newblue designs.  Ten teams.  

\subsection{Routability, 2011}

\input natarajan.tex
Details, cite some papers.

\subsection{Detail Routing Driven Placement, 2014, 2015}

Details, cite some papers.
Detail Routing.
{\bf Ismail?}



\subsection{FPGA Placement 2016, 2017}

Details, cite some papers.
{\bf Stephen?}


\subsection{Deep Learning Accelerator Placement, 2020}

Include this?  Not really placement, but placement-ish?


\section{The Benchmarks Ahead}

Striking a balance between objectives.  Industry groups
would certainly like a production-ready tool, but this is
something that a small team of graduate students can't
build quickly.

Need to capture the essence of a problem, while keeping it
tractable.  Make the benchmark hard to game, so that solutions
actually resemble what we might want ``in practice.''

Trade-off on using library for parsing -- locks into a build
system, sometimes a language and set of tools.  Simple
file formats, by contrast, may lose the essential elements
that matter for an industrial design.

Evaluators, with painstaking detail, are important.

Complex tool flows for evaluation are trouble.







\subsection{The OpenROAD}

Talk about open source tool flow.  

\cite{Ajayi19}

% Section on RDF
\input jinwook.tex


\section{Concluding Comments}

\begin{quote}
It seems apparent that more benchmarks for layout synthesis are
needed, they should be more fully described, as discussed earlier in
the paper. In the opinion of the author, distributing pure
geometrical data is no longer sufficient for meaningful benchmarking
of algorithms used in real-life applications to produce working chip
layouts. The circuit's function as well as its speed and timing
requirements should be supplied together with technological data
required to verify the performance of the completed design. As an
unavoidable consequence, future benchmarks will have to come from
sources willing to put the information about logical functions of the
benchmark circuits in the public domain.
\end{quote}


Certainly, much has changed over the years.  Preas\cite{Preas87}
noted in 1987 that some circuits with over 10,000 cells were becoming
commonplace -- challenging to handle because representing them
``may require a substantial fraction of a megabyte for storage.''
Circuits of this size might seem absurdly small today -- but they
were in fact challenging.

It's easy to forget -- but not so long ago, the
world wide web was not ubiquitous. Getting access to benchmarks, even if they
were ``publicly available,'' required knowing who to ask, where to
look, and the navigation of anonymous FTP servers.  Disk space,
processor cycles, and memory, were all in short supply -- even with
access to benchmarks, it might not be possible to do anything with them.
In this era, nearly everyone has a smart phone with a blazingly fast
processor, nearly infinite disk space, and vast amounts of
RAM on their workstations.  Many of the engineers
working today started their careers with punch cards,
paper terminals, and, if they were lucky, floppy disks.



\balance
\bibliographystyle{unsrt}
\bibliography{unify}


\end{document}

